\documentclass[ngerman,12pt,titlepage]{article}
\usepackage{amsmath,amssymb,amstext}
\usepackage{float}
\usepackage{pdfpages}
\usepackage{dcolumn} %für tabelle
\usepackage[utf8]{inputenc} %Codierung 
\usepackage[T1]{fontenc}
\usepackage{lmodern}
\usepackage[ngerman]{babel}
\usepackage{graphicx}
\usepackage{color}
\usepackage[colorlinks=true, linkcolor={black}]{hyperref}

\begin{document}
\pagenumbering{arabic}
\section{Definition der Äquivalenzklassen}

	\subsection{Übersicht der möglichen Nutzereingaben}
	
		\begin{table}[H]
			\centering
			\begin{tabular}{c|c|c|c}
				\textbf{Feldname}&\textbf{Wertebereich} & \textbf{Feldart} &\textbf{Repräsentant} \\  
				\hline
							  	&  [A-Z, a-z, 0-9, \_, -]			 	&&\\ 
				Nutzername		& \(5 \leq \) Namenslänge \(\leq 45\) 	&verpflichtend&awesome\_user-name\\ \hline
								& [A-Z, a-z, 0-9] \& Sonderzeichen 		&&\\  
				Passwort		& \(8 \leq \) Passwortlänge \(\leq 32\)	&verpflichtend&!PaS~ßw0rt?\\\hline
								& z \(\in\)[A-Z, a-z, 0-9, \_, -] 		&&\\
				%				& x \(\in\) [A-Z, a-z, 0-9]
				Email			& z*@z*.[a-z, A-z]* \textit{(*: mind 1\(\times\))} 		& verpflichtend&test\_email@test.uk\\
								& \(8 \leq \) Emaillänge \(\leq 320\) 	&&\\\hline
								& [0-9, +]								&&\\
				Telefonnummer	&\(8 \leq \) Nummerlänge \(\leq 15\)	&freiwillig&+49 0123 4567\\
			\end{tabular}
			\caption{Nutzereingaben der Registrierung}
			\label{tab:eingabe-nutzerregistrierung}
		\end{table}
		\noindent
		
		\begin{table}[H]
			\centering
			\begin{tabular}{c|c|c|c}
				\textbf{Feldname}&\textbf{Wertebereich} & \textbf{Feldart} & \textbf{Repräsentant}  \\  
				\hline
				Gruppenname  			& [A-Z, a-z, 0-9, \_, -] 				& &\\ 
										& \(8 \leq \) Namenslänge \(\leq 320\) 	&verpflichtend &coolest\_Group-4ever\\ \hline 
				Gruppenbild				& 0MB \(\leq\) Bildgröße \(\leq\) 5MB	&&\\  
										& Format png/jpg						&freiwillig&png der Größe 3MB\\ \hline
				Gruppenmitglieder		& \(1 \leq \) Mitgliederanzahl			&Suchfeld, verpflichtend&\\

			\end{tabular}
			\caption{Nutzereingaben der Gruppenerstellung}
		\end{table}
		\noindent
		
				\begin{table}[H]
			\centering
			\begin{tabular}{c|c|c|c}
				\textbf{Feldname}&\textbf{Wertebereich} & \textbf{Feldart} & \textbf{Repräsentant}  \\  
				\hline
				Turniername  			& [A-Z, a-z, 0-9, \_, -] 	&&\\ 
										& mind. 5 Zeichen 			&verpflichtend& great\_est-Game2020\\ \hline 
				Anzahl Turniermitglieder&  \(1 \leq \) Mitgliederanzahl	&&\\
				
			\end{tabular}
			\caption{Nutzereingaben der Turniererstellung}
		\end{table}
		\noindent

	\subsection{Ableitung von Äquivalenzklassen}
		\subsubsection{Äquivalenzklasse Gruppenname und zugehörige Testfälle}
		\begin{table}[H]
		\centering
		\begin{tabular}{l|c|c}
			\textbf{Äquivalenzklasse \textit{Gruppenname}} &\textbf{Wert für Feld}  &\textbf {Erwartetes Resultat} \\  
			\hline
			 \(8 \leq \) Zeichen \(\leq 320\) aus [A-Z, a-z, 0-9, \_, -]  & awesome\_user-name &gültig  \\  
			keine Eingabe (Zeichenanzahl = 0)&(leer)  & ungültig \\  
			Zeichenanzahl \(\leq\) 7 & 1234567 & ungültig \\  
			321 \(\leq\) Zeichenanzahl  & xx...x (Länge 46) &ungültig  \\  
		\end{tabular}
		\caption{Testfalltabelle der Äquivalenzklasse Gruppenname}
		\end{table}
	\noindent
	\subsubsection{Äquivalenzklasse Gruppenbild und zugehörige Testfälle}
			\begin{table}[H]
		\centering
		\begin{tabular}{l|c|c}
			\textbf{Äquivalenzklasse \textit{Gruppenbild}} &\textbf{Wert für Feld}  &\textbf {Erwartetes Resultat} \\  
			\hline
			Bild in Format jpg/png, Bildgröße \(\leq\) 5MB & png-Bild, Größe 3MB &gültig  \\  
			keine Eingabe &(leer)  & gültig \\  
			Bild nicht in Format jpg/png & 3MB xml Datei & ungültig \\  
			Bild in Format jpg/png, 5MB < Bildgröße  & jpg-Bild, Größe 5.1MB &ungültig  \\  
		\end{tabular}
		\caption{Testfalltabelle der Äquivalenzklasse Gruppenbild}
	\end{table}
	\noindent
	\subsubsection{Äquivalenzklasse Gruppenmitglieder und zugehörige Testfälle}
		\begin{table}[H]
		\centering
		\begin{tabular}{l|c|c}
			\textbf{Äquivalenzklasse \textit{Gruppenmitglieder}} &\textbf{Wert für Feld}  &\textbf {Erwartetes Resultat} \\  
			\hline
			Anzahl Gruppenmitglieder \(\leq\) 1 & Mitgliedsliste der Länge 3 &gültig  \\  
			keine Eingabe &(leer)  & ungültig \\    
		\end{tabular}
		\caption{Testfalltabelle der Äquivalenzklasse Gruppenmitglieder}
	\end{table}
	\noindent
\end{document}