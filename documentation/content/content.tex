\catcode`\_=13 
\def_{\textunderscore}
\catcode`_=8

\chapter{Thema des Projekts}
Das Ziel des Projekts ist die Entwicklung einer Anwendung zum Verwalten des Alltags in einer WG.\\
Weiterer Bestandteil der Anwendung ist eine "Tischkicker-App".\\\\
Zur weiteren Organisation des Projekts wurden User Stories und Tasks formuliert und geschätzt. Diese sind \href{https://studentdhbwmannheimde-my.sharepoint.com/personal/s181108_student_dhbw-mannheim_de/_layouts/15/onedrive.aspx?originalPath=aHR0cHM6Ly9zdHVkZW50ZGhid21hbm5oZWltZGUtbXkuc2hhcmVwb2ludC5jb20vOmY6L2cvcGVyc29uYWwvczE4MTEwOF9zdHVkZW50X2RoYnctbWFubmhlaW1fZGUvRXNFbnRKWC11VkpHa1N5MFZySWVxaHdCV2pQcl9MMzg3c0h3UUZvODJhZFVJZz9ydGltZT1xYkhMZm9yYzEwZw&viewid=7833eb22%2Db78c%2D4ca2%2D847f%2Dce9bd5826118&id=%2Fpersonal%2Fs181108%5Fstudent%5Fdhbw%2Dmannheim%5Fde%2FDocuments%2FVorlesungen%2F3%2E%20Semester%2FSoftware%2DEngineering%2FProjekt%2FArchiv}{hier}
zu finden.

\chapter{Arbeit am Projekt}
Hier werden jede Woche die getroffenen Beschlüsse der Gruppenarbeit festgehalten.
\section{Design}
\subsection{Beschlüsse aus dem 3. Semester}
Als Farben für die Oberfläche wurden schwarz und türkis gewählt.
	\begin{figure}[h!]
		\centering
		\includegraphics[width=0.5\linewidth]{text/Farben_des_Designs}
		\caption{Farbdesign}
		\label{farbdesign}
	\end{figure}
\subsection{Di, 07.04.2020}
Die Darstellung der Datenbank übernehmen das Design- und das Organisationsteam gemeinsam.\\
Des weiteren, soll jedes Mitglied des Designteams - ausgehend vom Farbdesign aus dem dritten Semester (Abbildung \ref{farbdesign}) - eine Startseite entwerfen, über diese wird in der nächsten Vorlesung entschieden.
Von dieser ausgehend werden dann die weiteren Ansichten entworfen.\\
Zum Design von Entwürfen soll \href{https://www.figma.com/}{figma.com} verwendet werden.
\subsection{Di, 14.04.2020}
	\begin{figure}[h!]
		\centering
		\includegraphics[width=0.4\linewidth]{text/Startseite}
		\caption{Startseite}
		\label{startseite}
	\end{figure}
Das abgebildete Schema für die Startseite (Abbildung \ref{startseite}) wurde mehrheitlich beschlossen. Da dieses aber noch nicht dem eigentlichen Farbdesign entspricht, wird dieses bis zur nächsten Woche noch einmal überarbeitet.\\
Neben der Festlegung auf dieses Schema für die Startseite wurden auf die Gruppenmitglieder der Design-Gruppe die Aufgaben verteilt, im Laufe der Woche bestimmte Ansichten zu entwerfen. Dazu gehören folgende Ansichten:\\
\begin{itemize}
	\item Erstellen einer Gruppe
	\item Anmelden
	\item Registrieren
	\item Erstellen eines Turniers
	\item Eintragen der Ergebnisse
	\item Statistiken
	\item Verwaltung des Profils
	\item Laufendes Turnier
\end{itemize}
\begin{figure}[h!]
		\centering
		\includegraphics[width=0.9\linewidth]{text/design_gridsystem}
		\caption{Gridsystem}
		\label{gridsystem}
	\end{figure}
Zusätzlich wurde für das Design ein Gridsystem (Abbildung \ref{gridsystem}) beschlossen. Demnach können in den Ansichten einzelne Zeilen in bis zu 12 Teile aufgeteilt werden, dies erleichtert eine einheitliche Gestaltung.
\subsection{Di, 21.04.2020}
Da die erstellten Designentwürfe noch keinem einheitlichen Schema entsprechen, wurden Angaben für die Überarbeitung der Designentwürfe der einzelnen Ansichten besprochen.
Zunächst wurde dazu eine svg-Datei als Template zur Verfügung gestellt am Beispiel der Ansicht des Testtuniers (Abbildung \ref{template}).\\
\begin{figure}[h!]
	\centering
	\includegraphics[width=0.6\linewidth]{text/Template}
	\caption{Template}
	\label{template}
\end{figure}
Zusätzlich wurden Farbwerte für die Designs in RGBA-Werten und die verwendete Schriftart angegeben:
\begin{itemize}
	\item Hintergrundfarben:
	\begin{itemize}
		\item Hellgrau: 2a333cff
		\item Dunkelgrau: 12141eff
		\item Türkis: 31bebeff
	\end{itemize}
	\item Schriftfarben:
	\begin{itemize}
		\item Weiß: ffffffff
		\item Türkis: 31bebeff
		\item Grau: bababaff
	\end{itemize}
	\item Eingabefeld:
	\begin{itemize}
		\item Schwarz: 000000ff
		\item Türkis: 31bebeff
	\end{itemize}
	\item Schriftart: sans-serif
\end{itemize}
Da auch noch einige Ansichten fehlten, werden von einzelnen Team-Mitgliedern Designs für folgende Ansichten entworfen:
\begin{itemize}
	\item Gruppe verwalten
	\item Letzte Spiele
	\item Beendetes Turnier
	\item Achievements
	\item FAQ
\end{itemize}
\subsection{Di, 28.04.20}
Der Entwurf der Datenbank (Abbildung \ref{DB}) wurde vorgestellt und besprochen.
\begin{figure}[h!]
	\centering
	\includegraphics[width=\linewidth]{text/Datenbankmodell}
	\caption{Datenbankentwurf}
	\label{DB}	
\end{figure}
\\Die finalen Designansichten sind nun \href{https://studentdhbwmannheimde-my.sharepoint.com/:f:/g/personal/s181108_student_dhbw-mannheim_de/Ek6DMqY0oCdFnB_9TndCsqAB3wNP20j2XAipOnYleT43Vw?e=0DY5WZ}{hier} einzusehen.
\\Für die Implementierung werden diese Designansichten in ein css-Format gebracht, damit das Coding-Team die Ansichten einbinden kann. Zusätzlich werden noch Designentwürfe mit dem dazugehörigen css-Format für folgende Ansichten entworfen:
\begin{itemize}
	\item Burgermenü
	\item Gruppenauswahl
	\item Spielerauswahl
	\item Bildkreis
	\item Spielzeile
	\item Alte Turniere
	\item Impressum
	\item Ansichten beim ersten Öffnen der App (Anmelden, Registrieren, Logo)
\end{itemize}



\newpage
\section{Coding}
\subsection{Di, 07.04.2020}
Als IDE soll Webstorm verwendet werden.\\
Ein \href{https://github.com/DavidRisch/kicker}{Github-Repository} wurde angelegt, dort sind im Wiki auch die Tasks, die im dritten Semester beschlossen wurden, zu finden.
\subsection{Di, 14.04.2020}
Die Struktur des Github-Repositorys ist wie folgt:
	\begin{itemize}
		\item \textit{api}: js files for backend post requests, returns json
		\item \textit{css}: all css files
		\item \textit{html}: all html files, mostly for use in \textit{pages/}
		\item \textit{js}: js files for use in the client browser
		\item \textit{page}: js files generating html on get requests
		\item \textit{src}: js files for importing in other javascript files
		
	\end{itemize}
\subsection{Di, 21.04.2020}
Erste Tasks, für die die Datenbank nicht nötig ist, wurden bereits bearbeitet.
Nach endgültiger Beschprechung der Datenbankstruktur, kann ab nächster Woche mit dem Hauptaufwand der Programmierung begonnen werden.
\subsection{Di, 28.04.2020}
Ab hier sind alle nötigen Vorraussetzungen für die Programmierung vorhanden und die Tasks werden abgearbeitet.
\newpage

\chapter{Use Cases und Aktivitätsdiagramme}
\section{Registrierung und Anmeldung}
\begin{figure}[h!]
	\centering
	\includegraphics{text/registrierung_anmeldung.png}
	\caption{Use Case: Registrierung und Anmeldung}
	\label{uc_reg_anm}
\end{figure}
\begin{figure}[h!]
	\centering
	\includegraphics[width=\linewidth]{text/aktiv_reg_anm.png}
	\caption{Aktivitätsdiagramm: Registrierung und Anmeldung}
	\label{ak_reg_anm}
\end{figure}
\textbf{Registrieren}
\begin{itemize}
	\item \textbf{Use Case:} Registrieren
	\item \textbf{Ziel:} Registriert
	\item \textbf{Kategorie:} Primär
	\item \textbf{Vorbedingung:} Nutzer hat bisher kein Konto
	\item \textbf{Nachbedingung Erfolg:} Konto angelegt
	\item \textbf{Nachbedingung Fehlschlag:} Konto nicht anlegbar
	\item \textbf{Akteure:} Nutzer
	\item \textbf{Auslösendes Ereignis:} Nutzer beendet Registrierungsprozess
	\item \textbf{Beschreibung:} \begin{enumerate}
		\item Eingegebenen Namen auf erlaubte Eingabewerte überprüfen
		\item Überprüfung, ob eingegebener Name bisher unbekannt ist
		\item E-Mail-Adresse auf erlaubte Eingabewerte überprüfen
		\item Überprüfung, ob eingegebene E-Mail-Adresse bisher unbekannt ist
		\item Eingegebenes Passwort auf erlaubte Eingabewerte überprüfen
		\item Telefonnummer auf reine Verwendung von Zahlen überprüfen
		\item Überprüfen, ob Datenschutzbedingungen akzeptiert wurden
		\item Eintragen der angegebenen Werte in die Datenbank
	\end{enumerate}
	\item \textbf{Erweiterungen:} /
	\item \textbf{Alternativen:} /
\end{itemize}


\textbf{Anmelden}
\begin{itemize}
	\item \textbf{Use Case:} Anmelden
	\item \textbf{Ziel:} Angemeldet
	\item \textbf{Kategorie:} Primär
	\item \textbf{Vorbedingung:} Nutzer hat bereits ein Konto
	\item \textbf{Nachbedingung Erfolg:} Nutzer ist am Server authentifiziert
	\item \textbf{Nachbedingung Fehlschlag:} Nutzer konnte nicht am Server authentifiziert werden
	\item \textbf{Akteure:} Nutzer
	\item \textbf{Auslösendes Ereignis:} Nutzer beendet Anmeldeprozess
	\item \textbf{Beschreibung:} \begin{enumerate}
		\item Nutzer an eingegebenem Namen bestimmen
		\item Nutzer mit eingegebenem Passwort identifizieren
		\item Nutzer auf Startseite weiterleiten
	\end{enumerate}
	\item \textbf{Erweiterungen:} /
	\item \textbf{Alternativen:} 1a. Nutzer an eingegebener E-Mail-Adresse bestimmen
\end{itemize}
\clearpage


\section{Turniererstellung}
\begin{figure}[h!]
	\centering
	\includegraphics[width=\linewidth]{text/uc_turnier.png}
	\caption{Use Case: Turniererstellung}
	\label{uc_turnier}
\end{figure}
\begin{figure}[h!]
	\centering
	\includegraphics[width=\linewidth]{text/ak_turnier.png}
	\caption{Aktivitätsdiagramm: Turniererstellung}
	\label{ak_turnier}
\end{figure}
\begin{itemize}
	\item \textbf{Use Case:} Turnier erstellen
	\item \textbf{Ziel:} Erstellen eines Turniers und Wechsel zur Turnieransicht
	\item \textbf{Kategorie:} Primär
	\item \textbf{Vorbedingung:} Nutzer hat ein Konto und ist angemeldet
	\item \textbf{Nachbedingung Erfolg:} Turnieransicht wird geöffnet und die angegebenen Teilnehmer werden benachrichtigt
	\item \textbf{Nachbedingung Fehlschlag:} Turnier wird nicht erstellt, es wird eine erneute Eingabe der Daten gefordert
	\item \textbf{Akteure:} Nutzer
	\item \textbf{Auslösendes Ereignis:} Nutzer klickt auf Link "Turnier erstellen" in der Seitenleiste oder auf der Startseite
	\item \textbf{Beschreibung:} \begin{enumerate}
		\item Eingabe eines Namens für das Turnier
		\item Auswahl des Spielmodus
		\item Auswahl des Turniermodus
		\item Auswahl der Mitspieler
		\item Absenden der Eingaben führt zur Erstellung des Turniers
	\end{enumerate}
	\item \textbf{Erweiterungen:} /
	\item \textbf{Alternativen:} Die Ausführungsschritte 2.-4. können in beliebiger Reihenfolge ausgeführt werden
\end{itemize}

\section{Spielerstellung}
\begin{figure}[h!]
	\centering
	\includegraphics{text/uc_spiel.png}
	\caption{Use Case: Spielerstellung}
	\label{uc_spiel}
\end{figure}
\begin{figure}[h!]
	\centering
	\includegraphics[width=\linewidth]{text/aktiv_spiel.png}
	\caption{Aktivitätsdiagramm: Spielerstellung}
	\label{aktiv_spiel}
\end{figure}
\begin{itemize}
	\item \textbf{Use Case:} Spiel erstellen
	\item \textbf{Ziel:} Spiel erstellt
	\item \textbf{Kategorie:} Primär
	\item \textbf{Vorbedingung:} Nutzer ist angemeldet und befindet sich in einer Gruppe und alle Mitglieder bei diesem Spiel haben einen Account und befinden sich in der selben Gruppe
	\item \textbf{Nachbedingung Erfolg:} Spiel wurde erstellt
	\item \textbf{Nachbedingung Fehlschlag:} Spiel konnte nicht erstellt werden
	\item \textbf{Akteure:} Nutzer
	\item \textbf{Auslösendes Ereignis:} Nutzer betätigt den "neues Spiel erstellen"-Button
	\item \textbf{Beschreibung:} \begin{enumerate}
		\item Spieler 1 von Team A aus dem Menü auswählen
		\item Anzahl Tore von Team A eintragen
		\item Spieler 1 von Team B aus dem Menü auswählen
		\item Anzahl Tore von Team B eintragen
		\item Spiel in der Datenbank hinzufügen
	\end{enumerate}
	\item \textbf{Erweiterungen:} \\1a. Spieler 2 von Team A aus dem Menü auswählen\\
	3a. Spieler 2 von Team B aus dem Menü auswählen
	\item \textbf{Alternativen:} /
\end{itemize}

\section{Gruppenerstellung}
\begin{figure}[h!]
	\centering
	\includegraphics[width=\linewidth]{text/uc_ak_gruppenerstellung.jpg}
	\caption{Use Case und Aktivitätsdiagramm: Gruppenerstellung}
	\label{uc_ac_gruppenerstellung}
\end{figure}
\begin{itemize}
	\item \textbf{Use Case:} Gruppenerstellung
	\item \textbf{Ziel:} Benutzer kann Gruppe mit mehreren Teilnehmern erstellen
	\item \textbf{Kategorie:} Primär
	\item \textbf{Vorbedingung:} Benutzer ist bereits registriert
	\item \textbf{Nachbedingung Erfolg:} Benutzer kann Gruppe verwalten
	\item \textbf{Nachbedingung Fehlschlag:} Keine Gruppe erstellt
	\item \textbf{Akteure:} Benutzer
	\item \textbf{Auslösendes Ereignis:} "Neue Gruppe erstellen" wurde ausgewählt
	\item \textbf{Beschreibung:} \begin{enumerate}
		\item Gruppennamen festlegen
		\item Gruppenmitglieder hinzufügen
		\item Gruppenbeschreibung festlegen
		\item Gruppenprofilbild festlegen
		\item Gruppenerstellung abschließen
		\item Gruppe in der Datenbank anlegen
		\item Einladungslink an Gruppenmitglieder versenden
		\item Gruppenansicht öffnen
	\end{enumerate}
	\item \textbf{Erweiterungen:} \\2a. Gruppenmitglieder löschen\\
	3a. Ohne Gruppenbeschreibung fortfahren\\
	4a. Ohne Gruppenbild fortfahren
	\item \textbf{Alternativen:} /
\end{itemize}

\section{Gruppenbeitritt}
\begin{figure}[h!]
	\centering
	\includegraphics[width=\linewidth]{text/uc_ak_gruppenbeitritt.png}
	\caption{Use Case und Aktivitätsdiagramm: Gruppenbeitritt}
	\label{uc_ac_gruppenbeitritt}
\end{figure}
\begin{itemize}
	\item \textbf{Use Case:} Gruppe beitreten
	\item \textbf{Ziel:} Erfolgreicher Gruppenbeitritt
	\item \textbf{Kategorie:} Primär
	\item \textbf{Vorbedingung:} Nutzer hat ein Konto
	\item \textbf{Nachbedingung Erfolg:} Beitrittsanfrage annehmen
	\item \textbf{Nachbedingung Fehlschlag:} Beitrittsanfrage ablehnen
	\item \textbf{Akteure:} Gruppenmitglied, neues Mitglied
	\item \textbf{Auslösendes Ereignis:} Nutzer erhält Einladungslink
	\item \textbf{Beschreibung:} \begin{enumerate}
		\item Einladungslink öffnen
		\item Beitrittsanfrage bestätigen
		\item Eintrag in Datenbank vornehmen
		\item Gruppenansicht anzeigen
		\item Startseite anzeigen
	\end{enumerate}
	\item \textbf{Erweiterungen:} 1a. Auf Kicker-Website anmelden\\
	2b. Beitrittsanfrage ablehnen
	\item \textbf{Alternativen:} /
\end{itemize}
\newpage
\chapter{Anhang}
\section*{Tasks}
Basis\\
\begin{tabular}[h]{|p{1cm}|p{10cm}|p{3cm}|}
\hline 
ID & Beschreibung & Dauer in Stunden \\ \hline 
1.1 & Aussuchen eines DBMS und Einrichten & 1 \\ \hline 
1.2 & Aussuchen eines Webservers für die notwendigen Funktionalitäten unter Berücksichtigung der Kosten & 0,5\\ \hline 
1.3 & Einrichten eines Webservers für die notwendigen Funktionalitäten & 0,5\\ \hline 
1.4 & Erstellen eines Impressums und einer Datenschutzerklärung & 5\\ \hline 
1.5 & Erstellen eines Favicons bzw. App-Icons & 7\\ \hline 
1.6 & Farbkonzept und Designentwurf erstellen (Startseite, Profil, Einstellung, Impressum, Datenschutzerklären) & 10\\ \hline 
1.7 & Navigationsleiste zur Bedienung & 2\\ \hline 
1.8 & Anmeldestatus bei jeder Aktion überprüfbar & 2\\ \hline 
1.9 & Auswählen der Gruppe durch den Benutzer & 2\\ \hline 
\end{tabular} \\ \\
Erstellen eines Kontos\\
\begin{tabular}[h]{|p{1cm}|p{10cm}|p{3cm}|}
\hline 
ID & Beschreibung & Dauer in Stunden \\ \hline 
2.1 & Auswahl zwischen Einloggen oder Konto erstellen & 0,5\\ \hline
2.2 & Erstellen der Eingabefelder ((Benutzer-)Name, Email, Passwort, Telefonnummer) & 1\\ \hline
2.3 & Erstellen eines Datenmodells & 2\\ \hline
2.4 & Überprüfung auf Eindeutigkeit des Benutzernamens & 1\\\hline
2.5 & Überprüfung von Passwortkriterien, Email und Namenskonventionen & 2\\\hline
2.6 & Benutzerdaten in Datenbank eintragen & 0,5 \\\hline
2.7 & Erstellen einer eindeutigen Benutzer-ID & 1\\\hline
2.8 & Erstellen einer Login-Maske & 1 \\\hline
2.9 & Anmeldung am Server & 1,5 \\\hline
\end{tabular}\newpage
Spielerergebnis eintragen für einzelne Spiele\\
\begin{tabular}[h]{|p{1cm}|p{10cm}|p{3cm}|}
\hline 
ID & Beschreibung & Dauer in Stunden \\ \hline 
3.1	& Datenmodell um Spieler und Turniere zu modellieren	& 8\\ \hline
3.2	& Einzelspiele werden erst nach dem Stattfinden erzeugt, wobei das Ergebnis gleich eingetragen wird & 2\\	 \hline
3.3	& Jedes Spiel wird mit Zeitstempel in der DB gespeichert & 0	\\ \hline
3.4	& Wertüberprüfung der Spieler und des Ergebnisses	& 2	\\ \hline
3.5	& Dropdown-Menü zum Auswählen der Spieler (keine Suchoption) & 3	\\ \hline
3.6	& Design der UI zum Eintragen eines Spiels und der Spieler & 3\\ \hline	3.7	& Implementierung der UI zum Eintragen eines Spiels & 4\\ \hline
\end{tabular}\\ \\
Es gibt 1:1 und 2:2 Spiele und 1:2/2:1\\
\begin{tabular}[h]{|p{1cm}|p{10cm}|p{3cm}|}
\hline 
ID & Beschreibung & Dauer in Stunden \\ \hline 
4.1 & UI für die Auswahl der Spieler pro Team implementieren	& 1 \\ \hline
\end{tabular}\\ \\
Konto nachträglich bearbeiten \\
\begin{tabular}[h]{|p{1cm}|p{10cm}|p{3cm}|}
\hline 
ID & Beschreibung & Dauer in Stunden \\ \hline 
5.1	& Design der UI zum Ändern der Benutzerdaten & 2 \\ \hline
5.2	& Implementierung der UI zum Ändern der Benutzerdaten & 2 \\ \hline	
5.3	 & Wertüberprüfung der Benutzerdaten & \\ \hline
5.4	 & Aktualisierung der Daten in DB & 1 \\ \hline
\end{tabular}\newpage
Gruppen erstellen; Nutzer sollen sich gegenseitig zu Gruppen hinzufügen, Anfragen verschicken ob man Mitglied werden darf \\
\begin{tabular}[h]{|p{1cm}|p{10cm}|p{3cm}|}
\hline 
ID & Beschreibung & Dauer in Stunden \\ \hline 
6.1	& Erstellung eines Datenmodells einer Gruppe & 2 \\ \hline
6.2	& Modell für Einladungslinks bzw. Beitrittsanfragen erstellen (universellen Link generieren) & 5	 \\ \hline
6.3 & Design eines Formulars zu Erstellung einer Gruppe	& 2 \\ \hline	
6.4	& Implementierung eines Formulars zu Erstellung einer Gruppe & 2 \\ \hline
6.5 & Gruppenansicht implementieren (Gesamt- und Einzelansicht) & 10	\\ \hline
6.6 & Nach Aufruf des Links wird der Benutzer der Gruppe hinzugefügt	& 2 \\ \hline
6.7 & Der Nutzer erhält eine Beitrittsbestätigung & 15\\ \hline
\end{tabular}\\ \\
Gruppen verlassen\\
\begin{tabular}[h]{|p{1cm}|p{10cm}|p{3cm}|}
\hline 
ID & Beschreibung & Dauer in Stunden \\ \hline 
7.1	& In der Gruppeneinzelansicht soll das Verlassen der Gruppe möglich sein $\rightarrow$ Button	& 2\\ \hline
7.2 & Bestätigungsformular vor dem Verlassen & 2 \\ \hline
7.3	& Implementierung einer Gruppenauflösung nach Verlassen des letzten Mitglieds & 1\\ \hline
7.4 & Daten einer Gruppe in DB aktualisieren & 0,5\\ \hline
\end{tabular}\newpage
Mehrere Turniermodi\\
\begin{tabular}[h]{|p{1cm}|p{10cm}|p{3cm}|}
\hline 
ID & Beschreibung & Dauer in Stunden \\ \hline 
8.1	& Ansicht zum Erstellen eines Turniers mit einem Menü zum Auswählen des Turniermodus und der Teamanzahl und Festlegung der Spieler und des Turniernamens & 13\\ \hline
8.2	& Erstellung eines Datenmodells für ein Turnier auf Basis des Gruppensystems & \\ \hline
8.3	& Alle Turnierteilnehmer sehen aktuelles Spiel auf Startseite und können dieses bearbeiten & 20 \\ \hline
8.4 & Ansicht mit offenen Spielen, Rangliste und Status (\%) vom Turnier & 15 \\ \hline	
8.5 & Design der UI zur Bearbeitung des Turniers & 8 \\ \hline
8.6 & Turniernamen festlegen		Dopplung & \\ \hline
8.7 & Turniermodi: Alle gegen Alle, KO-System, Schweizer-System & 30 \\ \hline	
8.8 & Nach Beendigung des letzten Spiels wird eine Auswertung angezeigt (Design) & 2 \\ \hline	
8.9 & Nach Beendigung des letzten Spiels wird eine Auswertung angezeigt (Implementierung) & 5 \\ \hline
\end{tabular}\\ \\
Aufruf eines Verlaufs der bisherigen Spielerergebnisse\\
\begin{tabular}[h]{|p{1cm}|p{10cm}|p{3cm}|}
\hline 
ID & Beschreibung & Dauer in Stunden \\ \hline 
9.1 & Tabelle zum Anzeigen der Spielergebnisse & 5 \\ \hline	
9.2 & Spielergebnisse nach eigenen Spielen filtern (Dropdown) & 15	\\ \hline
9.3 & Eintrag zu einem Spiel soll Spieler, das Ergebnis, Zeitpunkt und Gruppe beinhalten, ggf Turnier-Name (Design der Spielansicht) & 2 \\ \hline	
9.4 & Aufruf der Turnier-Detailansicht nach Anklicken des Turniernamens im Einzelspiel & 0,5 \\ \hline
\end{tabular}\newpage
Erstellen einer Statistik (Hall of Fame und Shame) wie viele Spiele wurden gewonnen und verloren\\
\begin{tabular}[h]{|p{1cm}|p{10cm}|p{3cm}|}
\hline 
ID & Beschreibung & Dauer in Stunden \\ \hline 
10.1 & Tabelle mit Plazierung, Benutzername, Anzahl der gewonnen und verlorenen Spiele (absolut und prozentual) mit Hervorhebung des eigenen Datensatzes & 30 \\ \hline
\end{tabular}\\ \\
Löschen der bisherigen Spiele\\
\begin{tabular}[h]{|p{1cm}|p{10cm}|p{3cm}|}
\hline 
ID & Beschreibung & Dauer in Stunden \\ \hline
11.1 & Nur Löschen des letzten Eintrags bei Turnier, Einzelspiele können jederzeit gelöscht und neu erstellt werden & 15 \\ \hline
11.2 & Formular zum Löschen und ein weiteres zum Bestätigen des Löschvorgangs eines Spiels & 3 \\ \hline
\end{tabular}\\ \\
Personalisierte Statistik gegen eine bestimmte Person nur gegeneinander 1:1\\
\begin{tabular}[h]{|p{1cm}|p{10cm}|p{3cm}|}
\hline 
ID & Beschreibung & Dauer in Stunden \\ \hline
12.1 & Auswahl eines Spielers in Statistikansicht für Einzelstatistik - Design & 2 \\ \hline	
12.2 & Auswahl eines Spielers in Statistikansicht für Einzelstatistik - Implementierung & 4 \\ \hline
\end{tabular}\\ \\
Einführen ELO Systems um Punkte eines Spielers zu berechnen\\
\begin{tabular}[h]{|p{1cm}|p{10cm}|p{3cm}|}
\hline 
ID & Beschreibung & Dauer in Stunden \\ \hline
13.1 & Datenmodell zur Speicherung des Verlaufs des ELO-Wertes & 0,5 \\ \hline	
13.2 & Nach Abschluss eines Spiels wird ELO-Wert aller Mitspieler neu berechnet und in DB angehängt & 6	\\ \hline
13.3 & aktuellen ELO-Wert in DB speichern (gesamten Verlauf speichern) & 0,5 \\ \hline	
\end{tabular}\\ \\
Rangliste der ELO Punkte und Spieler\\
\begin{tabular}[h]{|p{1cm}|p{10cm}|p{3cm}|}
\hline 
ID & Beschreibung & Dauer in Stunden \\ \hline
14.1 & Anzeigen einer Liste der ELO-Werte & 5 \\ \hline
\end{tabular}\newpage
Einführung von Achievements (Spiele ungeschlagen)\\
\begin{tabular}[h]{|p{1cm}|p{10cm}|p{3cm}|}
\hline 
ID & Beschreibung & Dauer in Stunden \\ \hline
18.1 & Achievements entwerfen (Bronze, Silber, Gold?) & 20 \\ \hline
18.2 & Erreichbarkeit über die benutzerspezifische Statistik & 1 \\ \hline
18.3 & Datenmodell entwerfen zum Speichern der Achievements & 3 \\ \hline
18.4 & Anzeige der Achievements - Implementierung & 15 \\ \hline
18.5 & Überprüfung der Kriterien der Achievements & 30 \\ \hline
\end{tabular}\\ \\
Mini-Chatraum für kleinere Nachrichten; Chat innerhalb einer Gruppe\\
\begin{tabular}[h]{|p{1cm}|p{10cm}|p{3cm}|}
\hline 
ID & Beschreibung & Dauer in Stunden \\ \hline
15.1 & Ein Eingabefeld implementieren & 2 \\ \hline
15.2 & Datenmodell zur Speicherung der Nachrichten & 4 \\ \hline
15.3 & Auf der Startseite soll eine Nachrichtenbenachrichtigung angezeigt werden & 5 \\ \hline
15.4 & Absende-Button & 2 \\ \hline
15.5 & Nachrichtenverlauf speichern & 1 \\ \hline
15.6 & Name des Nachrichtensenders wird neben/über Nachricht angezeigt mit Timestamp & 2 \\ \hline
15.7 & Neue Achivements werden im Mini-Chatraum verkündet & \\ \hline
\end{tabular}\\ \\
Senden einer Klingel - oder Klopf Nachricht\\
\begin{tabular}[h]{|p{1cm}|p{10cm}|p{3cm}|}
\hline 
ID & Beschreibung & Dauer in Stunden \\ \hline
16.1 & Funktion in Chat implementieren als Klingel-Button & 3 \\ \hline
16.2 & Push-Notification bei Empfang einer Klingelnachricht & 20 \\ \hline
\end{tabular}\\ \\
Erstellen einer FAQ Seite\\
\begin{tabular}[h]{|p{1cm}|p{10cm}|p{3cm}|}
\hline 
ID & Beschreibung & Dauer in Stunden \\ \hline
17.1 & Gliederung nach einzelnen Themen & 1	\\ \hline
17.2 & Aufarbeitung der Dokumentation zu den FAQ & 3 \\ \hline
\end{tabular}
\newpage
\includepdf[pages={1-}]{../images/text/Datenschutzerklarung}
	
\section*{Impressum}
Das Impressum wurde wie folgt festgelegt:\\\\
\textbf{Impressum}\\
Diese Website wurde im Rahmen der Vorlesung Software Engineering der DHBW Mannheim entwickelt.\\\\
Angaben gemäß §5 TMG\\\\
DHBW Mannheim\\
(Duale Hochschule Baden-Württemberg Mannheim)\\
Coblitzallee 1-9\\
68163 Mannheim\\\\
\textbf{Projektbetreuung}\\
Herr Schultheis\\\\
\textbf{Kontakt}\\
Telefon: +49 621 4105-0\\
E-Mail: info@dhbw-mannheim.de